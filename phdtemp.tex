%% Before beginning to type your dissertation, read the formatting guide, 
%% which can be found at http://grad.msu.edu/etd
%% and clicking Formatting Guide in the left hand column.
%% Also get the latest version of  msuphddissertation.cls and the template file
%% at http://www.math.msu.edu/~weil/MSU_Ph.D._Dissertation.zip
%% Send questions to weil@math.msu.edu

%%%%%%%%%%%%%%%%%%%%%%%%%%%%
%%%%%%%%  NOTE   %%%%%%%%%%%%%%
%% PREPARING A DISSERTATION WITH THIS CLASS FILE DOES NOT %%%
%% GUARANTEE THAT THE GRADUATE SCHOOL WILL APPROVE IT %%%
%%%%%%%%%%%%%%%%%%%%%%%%%%%%%%%
\RequirePackage{snapshot}
\documentclass{msuphddissertation2}
\usepackage{bm}% bold math
\usepackage{graphics}
\usepackage{amsmath}
\usepackage{graphicx}
\usepackage{epsfig}
\usepackage{longtable}
\usepackage{rotating}
\usepackage{listings}
\usepackage{color}
\pagecolor{white}
\usepackage[final]{pdfpages}
\newcommand{\be}{\begin{eqnarray}}
\newcommand{\ee}{\end{eqnarray}}
\newcommand{\Eref}[1]{Equation (\ref{#1})}
\newcommand{\Fref}[1]{Figure~\ref{#1}}
\newcommand{\Tref}[1]{Table~\ref{#1}}
\usepackage{multirow}
\usepackage{supertabular}
\newcommand{\plotlen}{4.4cm}
%% Insert packages you wish to use except setspace and subfig. 

\begin{document}

%\maketitlepage %%This command will produce the title page of your thesis.

%% If you wish to have a copyright page, remove the "%" in front of \begin{copyrt}
%% and remove the "%" in front of \end{copyrt}.
%% An acceptable form of a Copyright will be generated automatically
%% using the information you've already provided above. 
%% A copyright statement is optional, but you're strongly advised to include it.

%\begin{copyrt}
%\end{copyrt}

%% If you wish to have a dedication, remove the "%" in front of
%% \begin{dedication}
%% and remove the "%" in front of 
%% \end{dedication}
%% A dedication must be single-spaced and 
%% centered on the page.  Both will be done automatically. 

%\begin{dedication} 
%% Type your dedication here. A dedication is optional.
%\end{dedication}

%% If you wish to have an acknowledgment, remove the "%" in front of  \begin{acknowledgment}
%% and remove the "%" in front of  \end{acknowledgment}  
%\begin{acknowledgment}
%% Type your acknowledgment here. An acknowledgment is optional.
%\end{acknowledgment}

%% If you wish to have a preface, remove the "%" in front of \begin{preface}
%% and remove the "%" in front of \end{preface}. The formatting of
%% a preface isn't specified.
%\begin{preface}
%% Type your preface here. A preface is optional.
%\end{preface}

%\TOC %% This command produces the Table of Contents. DO NOT REMOVE!

%\tableofcontents
%\addtocontents{\TOC}{\singlespace}

\TOC
%% If your document contains tables, remove the "%" in front of 
%%  the following line.
\LOT
%% If your document contains figures, remove the "%" in front of
%% the following line.
\LOF

%%%% LIST OF SYMBOLS AND ABBREVIATIONS %%%%
%% If you wish to have a list of symbols, it should be here.
%% To create the list, remove the "%" in front of \begin{abbreviationskey}
%% and remove the "%" in front of \end{abbreviationskey}
%\begin{abbreviationskey}
%% Type your list as a list environment, a tabular environment or in any
%% fashion you wish here.
%\end{abbreviationskey}
%% The list will be included in the TOC as
%% KEY TO SYMBOLS AND ABBREVIATIONS
%% To change the name to something else, remove the "%" 
%% from the next line and change "DESIRED NAME" 
%% to your choice IN UPPER CASE.
%\renewcommand{\keyname}{DESIRED NAME}
%%%%%%%%%%

\newpage
\pagenumbering{arabic}



% not in use \input{Introduction}



\bibliographystyle{plain}

\bibliography{sample}

\end{document}

%%%%%% A FINAL COMMENT %%%%
%% Once your document has been filed with the Graduate School,
%% if you wish to produce a single spaced version of your document, 
%% find and remove the two commands \begin{doublespace}
%% and \end{doublespace} above.
