\section{Biological Introduction}

\subsection{Tree of Life and the Theory of Life}
The tree of life is both the organizing structure of life sciences (like it's predecessor- Linnean classification) and is a representation of the theory of life - evolution.  On a short time-scale it is a picture of biological reproduction - or unfaithful cloning - a pedigree.  On long time-scales, the tree is a representation of the process of species evolution, which in many ways can be thought of as a pedigree of species, where each node represents a population of a particular specie, and descent down the branches represents time, and division of a node into two nodes (reproduction in a standard pedigree - parents having children..) represents speciation events from a common ancestor (reproduction at a population level on geological time-scales).
\subsubsection{Comparative Anatomy and Physiology}
Previously,in the eighteenth century, Linnean by brute force clustering techniques organized life according to anatomical simililarites, leading to a classification system for life. This organization was not just a Victorian database that organized collections of living objects by comparative anatomy; the organization explained why the objects were different or why they were similar because comparative anatomy begs the question of 'function' of physiology, and physiology is a theory of life.   In this sense, Linnean systematic organization of life was a simple theory of life, in that it did explain life, as does all physiology, because it explained the purpose (function) of each anatomical feature\footnote{In biology the 'function' of a trait or anatomical structure is what the trait does or accomplishes, hence the word 'purpose' seems to be a synonym to function.  However, this may be confusing seeing that evolution has no 'purpose', hence 'purpose' should not be interpreted as if the population or lineage that evolved the trait had foresight and intended its creation.}.  For example, the purpose of a leg is locomotion, of a jaw is to bite, of a root is to stabilize a standing plant (among other functions), of a circulatory system (heart) is to circulate nutrients throughout the organism, of a stamen is for sexual reproduction etc.; and Linneaus knew these trivial relationships between structure and function, which undoubtedly helped in his grouping of organisms.  Interpreting the organization of life through a theory of structure and function is very powerful, as a theory intentional simplifies complex patterns so that they can be understood and comprehended.  Hence the theory of 'structure and function' has one of the most important aspects of biological theory, and that is to simplify life in a way that we can comprehend its rich diversity.  It also has an obvious predictive power, for example if an animal loses its legs you can predict that it will lose locomotion.  Although many of Linnean groups are based on reproductive anatomical features, a 'structure and function' theory is very short-sighted in terms of how life reproduces, as it can only explain how to maintain existing population of species (by having the reproductive structures do what they do).  What's clearly missing is the origin of life, how to make life from scratch; and the origin of all the different kinds or species (i.e. specie meaning a group that can form viable offspring).

The theory of evolution by Charles Darwin, which I'll summarize in two pieces 'descent' and 'modification' added  more structure to the theory of life.  Darwin recognized (hypothesized) that common anatomical structures were due to common 'descent', indicating common features need not be derived from scratch using vary different molecular ingredients, the entire structure was passed on during reproduction.  He also recognized that different structures between two groups of animals were due to 'modification' from a common ancestor between the two groups of animals.

Darwin's theory united life through one common lineage.  However the biological mechanisms of how to produce life from scratch (how are completely novel and complex features derived in the first place), or to show that one kind of organism could be related through another (through the common ancestor) were not sufficiently convincing to some people such that it be considered \textit{the} theory of life.  That would require the elucidation of two parts of molecular biology.  First, \textbf{molecular genetics} would show animals posses genes (which encode for traits) in common and that those genes were passed on from parents to children during reproduction, this culminated in the 'modern synthesis' of evolutionary biology.  The 'modern synthesis' is largely about the molecular dynamics of populations, 'population genetics', a population of organism's genomes (the frequency of genotypes) and how those change in time (in units of generations), and how they can be influenced by natural selection, mutation, gene flow, migration and isolation (E. Mayer type speciation).  The second part of molecular biology that supports macroevolution, and possibly an extension of the 'modern synthesis'\footnote{As suggested by the phrase, developmental genetics, this would seem to be a subdiscipline of 'molecular genetics' and hence not an extension of the modern sythesis, but rather a refining.  However, as I will discuss, the modern synthesis, was a gene centered theory.  It was about gene's that encode for proteins.  Developmental genetics is a gene 'regulatory' theory, it's all about the parts of the genome that turn traditional genes and and off; and gene regulatory elements are not really apart of the modern idea of a gene.  A gene encodes for a protein.  A genetic regulatory element encodes for something altogether different, and not a part of the modern synthesis.}, is \textbf{developmental genetics}, (this is still constroversial and ongoing) developmental genetics would show that master genes ( transcription factors) when their regulatory targets or binding sites evolved, then whole developmental networks (i.e. the genes necessary to build an anatomical structure) could be redeployed to a different position of a developing body, or lead to modifications of current body parts.  

Developmental genetics, many believe, is \textit{the} field of study that shows whole anatomical features can indeed be evolved, and that the diversity of anatomical features and their arrangenment (body plans) can be evolved.  This is because the detailed mechanisms that are needed to make Darwinian theory convincing at an anatomical scale (sometimes crudely called 'macroevolution') are now observed in the field of developmental genetics through powerful experiments such as transplantation experiments, and gain and loss of functions experiments, along with powerful tools molecular genetics tools and microscopy techniques and tools available to developmental geneticists.  In a sense developmental genetics, elucidates the 'Gene Reulatory Networks' that form the causal coarse grained molecular basis of self assembling anatomical features.  Hence, the molecular dynamics of how a fly is self assembled from a maternally laid egg, a single cell, is coarsly understood at the molecular level.  This does not explain 'macroevolution', anatomical evolution, however if gain or lose or modification of master genes (which encode transcription factors acting in early development) or of gene regulatory binding sites that interact with the master genes did result in modification or gain of loss of anatomical features, then this would explain the evolutionary mechanism of anatomical features\footnote{This type of huge leaps in modification (leaps) of an organism was observed in the fossil record by Jon Conway (among others), a paleontologist, which he called the Cambrian Explosion, and is in contrast (constrast does not mean conflict or contradictory) to the ubiquitously accepted 'gradual accumulation' of beneficial mutations that results in adabtations within a particular genera of organsisms (this was even known by plant and animal breeders before Darwin's theory).  The idea of saltation was called 'punctuated equilibrium' by Steven Jay Gould, where periods of gradual accumulation are punctuated by bursts of major anotomical features.}  

(i.e. transcription factors that regulate other transcription factors that ultimately feed in to signalling networks and paracrine factors that lead to the construction of anatomical features). built from molecule dynamics and population genetics, and hence answers the question of how to build 'from scratch' a body plan (the arrangement of anatomical features)\footnote{'From scratch' is misleading when discussing body plans of animals, as the diversity of all multicellular organisms is not due to building from scratch (i.e. independent evolution such as convergence or parallelisms), body plans are thought to derive through the reconstructions of a set of modules (devolopmental networks of genes) that each encode for anatomical features (hence all eyes, for example, are homologies at a developmental level through the master gene (transcription factor) \textit{pax} that binds to a set of genes that set a 'totipotent' or partially programmed cell to start to 'develop' the eye imaginal disc (the set of cells that form the basic outline of an eye, which when induced, will form an eye (at any part of a body that 'induction' occurs (even in the tail region if so induced).  The idea of a totipotent cell, and cell specification, are necessary to explain development, and hence necessary component of an extension to the modern sythesis.  Cell differentiation and cell lineages I will briefly discuss below.).  Of course, the initial body with many of the modules (urbiliteria, the ancestor of all animals) or its fungal and then algal ancestor, still needs to be explained as how one builds it from scratch, but must people are satisfied if urbiliteria can explain the diversity of animals.  Furthermore, the statement that anatomical features are modular at the genetic level is not obvious, as many believe complex features (anatomical structures) contain highly correlated genetic networks, such that any genetic mutation would be deleterious and probably lethal.  The discovery of the extent of modularity (or the extent of nonmodularity through 'induction' by signaling and paracrine factors) I will discuss briefly in the section below on 'modularity'.}.        
%\subsubsection{Comparative Embryology}
\subsubsection{The origin of multicellularity; the precursors to development}
About a billion or two years ago, single celled algae started to cooperate by developing cell cell interactions, forming the first eukaryotic multicellular  organisms, like algal mats.  Such constructs eventually lead to the important innovation of multicellular organisms that is found in animals (and other kingdoms): sexual reproduction through meiosis and fertilization.  Like endosymbiosis, which is possibly the origin of eukaryoutes from bacteria, where two cells would merge and partner to share their particular genes and hence traits (which may be different traits), in sexual reproduction cells not only merge, or fertilize (merging of two gametes into a zygote), but they first go through a phase which is known as meiosis, which is significantly different from the results of endosymbiosis (which replicate through mitosis of each fused component) significantly different due to the 'recombination' of traits, or crossing over, leading to great diversity in progeny, thereby leading to faster evolution (by Fisher's Fundamental Theorem) and preventing Muller's ratchet (by creating gamete's free of detrimental mutations).  Furthermore, the fusion of two genomes (e.g. the two gamete genomes) is a form of 'gene duplication', which is a source in 'gene families', where the copied gene leads to diversity in function of family of genes\footnote{Two bacterial cells could fuse leading to a n -> 2n genotype, similar to diploidy, and allowing for greater genomic diversity as one of the duplicate copies of the gene are now possibly free to serve a new function.  However, this genome duplication event is distinct from sexual reproduction (and hence is not diploidy, in my opinion), as the new bacteria with 2n genes, can be said to now have simply n' genes (a haploid with n' genes), when the n' genes are replicated (through mitosis), the progeny are clones (if mutation rate is slow enough), which distinguishes it from sex, in sexual reproduction the progeny are not clones due to recombination (assuming the mutation rate is high enough that there exists some genetic polymorphisms in the population, such that the cross over pieces are not identical).}.  

These early sexual reproducing algea are the origin of 'eukaryotic early development' (which I define as eukaryotic cellular interactions that lead to a multicellular stucture, such as an algal mat (a plain of cells), or a 'blastula' (a ball of cells)\cite{pmid7579526}).  They are modern representatives, living fossils, of the evolution of multicellularity.  

In the diverse domain of eukaryote the most famous groups are multicellular organisms, due to their gross anatomical features, in these multicellular organisms there is a remarkable common or conserved early development, yet there is also marked differences suggesting that multicellularity has independently evolved in plants and animals and fungus.  

Examples of multicellularity can be even seen in bacteria in (quorom sensing) and the primitive fungus yeast through their form of sex using 'mating types' (analogs of males and females)\cite{wolpert}. An array of protists (eukaryotes that are not animals plants or fungus) show multicellularity, notably volvox and slime molds.  All these 'primitive' organisms are good starting points for the study of multicellularity.  However 'development', to a large degree, focuses on multi-cell types (a skin cell 'functions' differently than a germ cell and functions different than a stomach cell (digestive cell that can 'eat' absorbed surroundings ).  That is the division of labor through specialized cells, which is not seen in many primitive organisms, rather these simple organisms are displaying 'colonies', which are the aggregates of cells due to mitosis resulting in progeny cells being proximal to one another (which is physical important in development, but it doesn't display the division of the aggregate into different cell types).

The starting point of 'development' is the innovation of meiosis, invented by the protists, and passed on to its progeny lineages of plants animals and fungus.  Hence, plants and animals don't derive sex for themselves from scratch, they were the benefactories of it from a protistan ancestor.  By the union of meiosis with fertilization (the opposite of meiosis, in a sense) the ability to always have an extra copy of a gene, and therefore the ability of a gene to evolve a new function becomes a stable component of life forms (regardless of whether some life forms display a 'haploid dominant' life form, like ferns, where the plant one normally sees is the haploid, this is because most of the ferns \textit{life cycle} exist in the haploid stage (even when it's an 'adult')).


The great lineages of animals, fungus and plants all develop from the fertilized 'egg' (the fusion of the two gametes)\footnote{'Egg' means oocyte here, one of the gametes, while in development literature, egg may also mean the structure that encases a 'female's' offspring, like a chicken egg.  Initially in evolution, possibly, there was no distinction between gametes, egg and sperm, both gametes were equal, just haploid cells, like in yeasts (which don't go through meiosis, just fertilization).}. to a ball of cells blastula as observed by the earliest of embryologist Karl Ernst von Baer. , who systematically studied blastulas and conjectured that the embryo of some particular animal does not look like the adult of another different animal, was that the origin of a blastula isn't too far in time (possibly animals evolved in parallel with algal mats and early plants) and in space (just flatten the ball of cells into a plain of cells) from an algal mat.     



In the late nineteenth century,  and other classification schemes of life to evolutionary classification.  This shift was largely just an added layer of meaning of the Linnean database, as the database's groups, to a large extent, were kept in tact, as Linnean groups that shared characteristics were shared characteristics precisely because of evolution - common descent, and the differences betweens groups was because of evolution - descent with modification, speciation.    

The evidence that Linnean group's were the result of evolution is visually suggested by the observation that children are different than parents yet similar to parents.  The strongest evidence, recognized by Mendel, is based on the genetic (molecular) thoery of inheritance (i.e. the modern synthesis of evolution); the understanding of molecular replication of genomes (meiosis and mitosis, cell division) and its unfaithfulness; as children are not clones of a particular parent, albeit purebreeds can come close.  

About four billion years ago, the first population of replicators spawned the root of the tree of life, and genomes have been replicating with errors ever since.  The evidence for the root of the tree is based on the molecular data available through genomic DNA sequencing, where the first replicators are thought to be based on the enzymatic activity of the ribosome, a molecule encoded in a small network of genes that are \textit{conserved} between prokaryotes and the eukaryotes (the ribosome is an amalgam of protein and RNA, where the RNA is encoded in the genome (rRNA) - leading to the RNA-world origin of life hypothesis).

It is currently thought that the first two billion years of life was a progressive one, increasing complexity, as networks of genes necessary for functions like metabolism and photosynthesis and respiration were evolved.  Then, again progressively, it is thought multicellular organisms are thought to have evolved, leading to the dinosaurs, their meteoritic extinction and the aftermath of mammalian and other verterbrates and inverterbrates species expansion.

The amazing diversity of multicellular organisms, from dinosaurs to humans to a rose, was partly generated, of course, by standard molecular evolution ( mutation events of genes (that encode proteins) during replication may result in replacement of an amino-acid).  But standard molecular evolution, somewhat obviously, can't possibly explain the difference between a shark and a human (notice in Zukerlandyl and Paulie's famous plot of hemoglobin the ends of the linear plot were human and shark).  Yes, the hemoglobin in shark has adapted to the environment the shark resides in, leading to differences with human hemoglobin, but the shark is still vastly different in body structure.

The diversity of the animal kingdom can not be explained by standard molecular evolution; the differences between proteins in animals is not sufficient to explain the diversity in animal phenotypes\cite{pmid1090005}.  It is thought there is about 20,000 proteins, and these proteins (for example, hemoglobin) are largely conserved across the kingdom.  The diversity is now thought, to a large extent, to be due to the way these proteins are used in different amounts and in different combinations in different cells and body parts of different multicellular organism.  To change the amount or create combinations of various proteins in a particular cell in an organism is accomplished by gene regulation.  Hence, in short, the diversity is due to the evolution of the elements of gene regulatory networks, where the transcription factor binding site is THE fundamental unit.
\subsection{Evolution and Development}
The work of Ed Lewis on body-transformative genes that were 'saltationary'\footnote{Saltation is the idea that within just one generation major evolutionary transformations could occur, possibly even speciation, taking the idea to its fictional extreme would be like an ape having a baby human, a 'hopeful monster', hence in one generation a speciation event occurred.} helped provide evidence and a theoretical framework (extending evolution theory) for the statement that diversity in the animal kingdom (in particular the \textit{segmented} insects) is due to evolution of gene reguatory networks.  The "Hox" genes were a group of genes that caused 'Homeotic transformations', which means a transformation in a body part of an animal, transformations that were known to some biologist as early as the nineteenth century.  \textit{The} exemplary gene in this group is 'bithorax', which was discovered in Thomas Hunt Morgan's lab in 1915, and a lineage of these mutants has been preserved since then. The gene is named after its mutation that lead to a mutant fly with two (bi) thoraxes (like an abdomen), which consequently doubles the number of wings of the fly (since the thorax is where the wing's developmental source (or pool of cells that each have the right genes turned on and off- a form of programming - leads to the developmental source of cells of the wing, these 'programmed' cells during early development are the so-called 'imaginal disk', and in this case, the wing imaginal disk\footnote{place a 'wing imaginal disk' in the location of the head, and you get wings growing out of the head, or place a 'leg imaginal disk' - the cells 'programmed' to build a leg where a wing is supposed to occur and you'll see leg's 'grow' or further develop where the wing's were supposed to be.} occurs, and hence is a homeotic transformation.  The Morgan lab didn't know the molecular genetic basis behind bithorax (what DNA sequence had changed (or possibly networks of DNA sequences) from the wild type fly to the mutant), but they did know the mutation was inheritable, and hence was genetic.




\subsection{Multiple Sequence Alignment}
Conservation of genetic elements (conserved DNA sequences) is the basis of similarities in Linnean groups, in species.  Mutation of genetic elements is the basis of difference in species.  Conservation of genetic elements is as simple as comparing to words (sequences) by checking if they 'match' at each position.


 



  
\subsection{Evolution and Development}